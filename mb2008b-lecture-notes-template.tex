\documentclass[12pt]{article}

\usepackage{amsmath, amssymb, amsthm}
\usepackage{graphicx}
\usepackage{hyperref}
\usepackage{geometry}
\geometry{a4paper, margin=1in}

\title{Numerical Analysis Lecture Notes}
\author{Instructor: Juliho Castillo}
\date{\today}

\begin{document}

\maketitle
\tableofcontents
\newpage

\section{Introduction}
\subsection{What is Numerical Analysis?}
Numerical analysis is the study of algorithms that use numerical approximation for the problems of mathematical analysis.

\subsection{Applications}
Numerical analysis has applications in all fields of engineering and the physical sciences.

\section{Error Analysis}
\subsection{Types of Errors}
\begin{itemize}
    \item \textbf{Round-off Error:} Errors due to the finite precision of computer arithmetic.
    \item \textbf{Truncation Error:} Errors made when an infinite process is approximated by a finite one.
\end{itemize}

\subsection{Error Propagation}
Understanding how errors propagate through computations is crucial for the design of robust numerical algorithms.

\section{Numerical Methods}
\subsection{Root Finding}
\begin{itemize}
    \item \textbf{Bisection Method}
    \item \textbf{Newton's Method}
\end{itemize}

\subsection{Interpolation}
\begin{itemize}
    \item \textbf{Lagrange Interpolation}
    \item \textbf{Newton Interpolation}
\end{itemize}

\subsection{Numerical Integration}
\begin{itemize}
    \item \textbf{Trapezoidal Rule}
    \item \textbf{Simpson's Rule}
\end{itemize}

\section{Conclusion}
Numerical analysis is a vital field that bridges the gap between pure mathematics and practical computation.

\end{document}